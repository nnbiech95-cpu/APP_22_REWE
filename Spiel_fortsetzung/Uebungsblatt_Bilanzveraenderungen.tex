\documentclass[11pt,a4paper]{article}
\usepackage[utf8]{inputenc}
\usepackage[ngerman]{babel}
\usepackage[T1]{fontenc}
\usepackage{geometry}
\usepackage{graphicx}
\usepackage{tabularx}
\usepackage{booktabs}
\usepackage{enumitem}
\usepackage{xcolor}
\usepackage{amssymb}
\usepackage{tikz}
\usepackage{hyperref}
\usepackage{fancyhdr}

\geometry{a4paper, margin=2.5cm}

\pagestyle{fancy}
\fancyhf{}
\fancyhead[L]{\textbf{Übungsblatt: Bilanzveränderungen}}
\fancyhead[R]{\thepage}
\renewcommand{\headrulewidth}{0.5pt}

\definecolor{maincolor}{RGB}{102, 126, 234}
\definecolor{lightgray}{RGB}{248, 249, 250}

\title{\textbf{\huge Übungsblatt: Bilanzveränderungen}}
\author{}
\date{}

\begin{document}

\maketitle
\thispagestyle{fancy}

\noindent
\textbf{Name:} \rule{5cm}{0.4pt} \hfill \textbf{Datum:} \rule{3cm}{0.4pt}

\vspace{0.5cm}

\section*{🎯 Lernziele}
Nach diesem Übungsblatt kannst du:
\begin{itemize}[leftmargin=*]
    \item[$\checkmark$] Die 4 Arten von Bilanzveränderungen sicher erkennen
    \item[$\checkmark$] Geschäftsvorfälle richtig einordnen
    \item[$\checkmark$] Auswirkungen auf Aktiva und Passiva bestimmen
    \item[$\checkmark$] Bilanzsummen korrekt berechnen
\end{itemize}

\section*{Aufgabe 1: Wiederholung - Die 4 Bilanzveränderungen}

Erkläre kurz, was bei den folgenden Bilanzveränderungen passiert und gib ein Beispiel:

\vspace{0.3cm}

\begin{tabularx}{\textwidth}{|l|X|}
\hline
\textbf{Aktiv-Tausch} & \vspace{1cm} \\
\hline
\textbf{Passiv-Tausch} & \vspace{1cm} \\
\hline
\textbf{Bilanzverlängerung} & \vspace{1cm} \\
\hline
\textbf{Bilanzverkürzung} & \vspace{1cm} \\
\hline
\end{tabularx}

\section*{Aufgabe 2: Geschäftsvorfälle zuordnen}

Ordne die folgenden Geschäftsvorfälle der richtigen Bilanzveränderung zu!

\textbf{A} = Aktiv-Tausch \quad | \quad \textbf{P} = Passiv-Tausch \quad | \quad \textbf{V} = Bilanzverlängerung \quad | \quad \textbf{K} = Bilanzverkürzung

\vspace{0.3cm}

\begin{tabularx}{\textwidth}{|c|X|c|}
\hline
\textbf{Nr.} & \textbf{Geschäftsvorfall} & \textbf{Bilanzveränderung} \\
\hline
1 & Ein Unternehmen kauft einen Computer für 1.200 € und zahlt bar. & $\square$ A  $\square$ P  $\square$ V  $\square$ K \\
\hline
2 & Ein Lieferant wird per Banküberweisung bezahlt (800 €). & $\square$ A  $\square$ P  $\square$ V  $\square$ K \\
\hline
3 & Das Unternehmen nimmt einen Kredit über 15.000 € auf. Das Geld wird auf das Bankkonto überwiesen. & $\square$ A  $\square$ P  $\square$ V  $\square$ K \\
\hline
4 & Büromaterial wird auf Rechnung gekauft (300 €). & $\square$ A  $\square$ P  $\square$ V  $\square$ K \\
\hline
5 & Ein Kunde zahlt seine offene Rechnung (2.500 €) per Überweisung. & $\square$ A  $\square$ P  $\square$ V  $\square$ K \\
\hline
6 & Das Unternehmen tilgt einen Teil des Kredits (5.000 €). & $\square$ A  $\square$ P  $\square$ V  $\square$ K \\
\hline
7 & Ein langfristiger Kredit wird in kurzfristige Verbindlichkeiten umgewandelt (10.000 €). & $\square$ A  $\square$ P  $\square$ V  $\square$ K \\
\hline
8 & Waren werden auf Rechnung eingekauft (4.500 €). & $\square$ A  $\square$ P  $\square$ V  $\square$ K \\
\hline
9 & Der Inhaber zahlt Geld aus der Kasse auf das Bankkonto ein (1.000 €). & $\square$ A  $\square$ P  $\square$ V  $\square$ K \\
\hline
10 & Ein Darlehen wird durch Eigenkapital ersetzt (20.000 €). & $\square$ A  $\square$ P  $\square$ V  $\square$ K \\
\hline
11 & Ware wird verkauft und der Kunde zahlt bar (600 €). & $\square$ A  $\square$ P  $\square$ V  $\square$ K \\
\hline
12 & Ein Fahrzeug wird gekauft und per Kredit finanziert (25.000 €). & $\square$ A  $\square$ P  $\square$ V  $\square$ K \\
\hline
\end{tabularx}

\newpage

\section*{Aufgabe 3: Bilanzveränderungen berechnen}

\textbf{Ausgangsbilanz der Firma ``TechStart GmbH'':}

\vspace{0.3cm}

\begin{center}
\begin{tabular}{|l|r||l|r|}
\hline
\textbf{AKTIVA} & \textbf{€} & \textbf{PASSIVA} & \textbf{€} \\
\hline
Maschinen & 30.000 & Eigenkapital & 50.000 \\
Waren & 15.000 & Bankkredit & 20.000 \\
Forderungen & 10.000 & Verbindlichkeiten & 5.000 \\
Bank & 20.000 & & \\
\hline
\textbf{Summe} & \textbf{75.000} & \textbf{Summe} & \textbf{75.000} \\
\hline
\end{tabular}
\end{center}

\vspace{0.5cm}

\subsection*{Führe die folgenden Geschäftsvorfälle durch:}

\begin{enumerate}
\item \textbf{GV 1:} Waren kaufen (8.000 €, auf Rechnung) \quad $\rightarrow$ \rule{4cm}{0.4pt} \quad Bilanzsumme: \rule{2.5cm}{0.4pt}
\item \textbf{GV 2:} Kunde zahlt Rechnung (3.000 €, per Bank) \quad $\rightarrow$ \rule{4cm}{0.4pt} \quad Bilanzsumme: \rule{2.5cm}{0.4pt}
\item \textbf{GV 3:} Verbindlichkeiten zahlen (2.000 €, per Bank) \quad $\rightarrow$ \rule{4cm}{0.4pt} \quad Bilanzsumme: \rule{2.5cm}{0.4pt}
\end{enumerate}

\vspace{0.5cm}

\subsection*{Erstelle die Endbilanz nach allen drei Geschäftsvorfällen:}

\vspace{0.3cm}

\begin{center}
\begin{tabular}{|l|r||l|r|}
\hline
\textbf{AKTIVA} & \textbf{€} & \textbf{PASSIVA} & \textbf{€} \\
\hline
Maschinen & \rule{2cm}{0.4pt} & Eigenkapital & \rule{2cm}{0.4pt} \\
Waren & \rule{2cm}{0.4pt} & Bankkredit & \rule{2cm}{0.4pt} \\
Forderungen & \rule{2cm}{0.4pt} & Verbindlichkeiten & \rule{2cm}{0.4pt} \\
Bank & \rule{2cm}{0.4pt} & & \\
\hline
\textbf{Summe} & \rule{2cm}{0.4pt} & \textbf{Summe} & \rule{2cm}{0.4pt} \\
\hline
\end{tabular}
\end{center}

\newpage

\section*{Aufgabe 4: Fehler finden}

Folgende Geschäftsvorfälle wurden \textbf{falsch} zugeordnet. Finde die Fehler und korrigiere sie!

\vspace{0.3cm}

\begin{tabularx}{\textwidth}{|c|X|c|X|}
\hline
\textbf{Nr.} & \textbf{Geschäftsvorfall} & \textbf{Falsche Zuordnung} & \textbf{Richtige Bilanzveränderung} \\
\hline
1 & Kauf von Rohstoffen auf Rechnung & Aktiv-Tausch & \rule{3cm}{0.4pt} \\
\hline
2 & Kunde zahlt per Überweisung & Bilanzverlängerung & \rule{3cm}{0.4pt} \\
\hline
3 & Tilgung eines Kredits per Bank & Aktiv-Tausch & \rule{3cm}{0.4pt} \\
\hline
4 & Umwandlung von langfristigem in kurzfristigen Kredit & Bilanzverlängerung & \rule{3cm}{0.4pt} \\
\hline
5 & Kauf eines Computers per Banküberweisung & Bilanzverlängerung & \rule{3cm}{0.4pt} \\
\hline
\end{tabularx}

\section*{Aufgabe 5: Praxisfall - Möbelhaus Müller}

Das Möbelhaus Müller hat folgende Bilanz:

\vspace{0.3cm}

\begin{center}
\begin{tabular}{|l|r||l|r|}
\hline
\textbf{AKTIVA} & \textbf{€} & \textbf{PASSIVA} & \textbf{€} \\
\hline
Ladeneinrichtung & 40.000 & Eigenkapital & 60.000 \\
Waren & 25.000 & Bankkredit & 30.000 \\
Bank & 25.000 & Verbindlichkeiten & 10.000 \\
Kasse & 10.000 & & \\
\hline
\textbf{Summe} & \textbf{100.000} & \textbf{Summe} & \textbf{100.000} \\
\hline
\end{tabular}
\end{center}

\subsection*{Am Montag passieren folgende Geschäftsvorfälle:}

\begin{enumerate}[label=\alph*)]
\item Möbel kaufen (12.000 €, auf Rechnung) \quad $\rightarrow$ \rule{3.5cm}{0.4pt} \quad Waren: \rule{2cm}{0.4pt} € \quad Verb.: \rule{2cm}{0.4pt} € \quad BS: \rule{2cm}{0.4pt} €

\item Sofa verkaufen (3.500 €, bar) \quad $\rightarrow$ \rule{3.5cm}{0.4pt} \quad Kasse: \rule{2cm}{0.4pt} € \quad Waren: \rule{2cm}{0.4pt} € \quad BS: \rule{2cm}{0.4pt} €

\item Lieferant bezahlen (5.000 €, per Bank) \quad $\rightarrow$ \rule{3.5cm}{0.4pt} \quad Bank: \rule{2cm}{0.4pt} € \quad Verb.: \rule{2cm}{0.4pt} € \quad BS: \rule{2cm}{0.4pt} €

\item Kasse auf Bank einzahlen (2.000 €) \quad $\rightarrow$ \rule{3.5cm}{0.4pt} \quad Kasse: \rule{2cm}{0.4pt} € \quad Bank: \rule{2cm}{0.4pt} € \quad BS: \rule{2cm}{0.4pt} €
\end{enumerate}

\subsection*{Erstelle die Endbilanz nach allen Geschäftsvorfällen:}

\vspace{0.3cm}

\begin{center}
\begin{tabular}{|l|r||l|r|}
\hline
\textbf{AKTIVA} & \textbf{€} & \textbf{PASSIVA} & \textbf{€} \\
\hline
Ladeneinrichtung & \rule{2cm}{0.4pt} & Eigenkapital & \rule{2cm}{0.4pt} \\
Waren & \rule{2cm}{0.4pt} & Bankkredit & \rule{2cm}{0.4pt} \\
Bank & \rule{2cm}{0.4pt} & Verbindlichkeiten & \rule{2cm}{0.4pt} \\
Kasse & \rule{2cm}{0.4pt} & & \\
\hline
\textbf{Summe} & \rule{2cm}{0.4pt} & \textbf{Summe} & \rule{2cm}{0.4pt} \\
\hline
\end{tabular}
\end{center}

\section*{Aufgabe 6: Kreatives Denken}

Erfinde \textbf{eigene Geschäftsvorfälle} für ein fiktives Unternehmen deiner Wahl:

\vspace{0.3cm}

\begin{tabularx}{\textwidth}{|l|X|}
\hline
\textbf{Aktiv-Tausch} & \vspace{0.8cm} \\
\hline
\textbf{Passiv-Tausch} & \vspace{0.8cm} \\
\hline
\textbf{Bilanzverlängerung} & \vspace{0.8cm} \\
\hline
\textbf{Bilanzverkürzung} & \vspace{0.8cm} \\
\hline
\end{tabularx}

\section*{Bonusaufgabe: Expertenaufgabe}

Ein Unternehmen hat eine Bilanzsumme von 250.000 €. Nach mehreren Geschäftsvorfällen beträgt die Bilanzsumme nun 275.000 €.

\textbf{Frage:} Welche Arten von Bilanzveränderungen \textbf{können} stattgefunden haben? (Mehrere Antworten möglich)

$\square$ Nur Bilanzverlängerungen \quad
$\square$ Nur Bilanzverkürzungen \quad
$\square$ Bilanzverlängerungen und -verkürzungen (Verlängerungen überwiegen)

$\square$ Aktiv-Tausch und Passiv-Tausch \quad
$\square$ Alle 4 Arten von Bilanzveränderungen

\textbf{Begründung:} \rule{12cm}{0.4pt}

\vspace{1cm}

\section*{Tipps}

\textbf{1)} Achte auf Wörter wie ``bar'', ``auf Rechnung'', ``per Überweisung'' \quad
\textbf{2)} Nutze die App ``Lenas Café'' zur Visualisierung \quad
\textbf{3)} Die Bilanzsumme muss auf beiden Seiten immer gleich sein!

\vspace{1cm}

\noindent
\textbf{Abgabe bis:} \rule{5cm}{0.4pt}

\vspace{0.5cm}

\begin{center}
\textbf{\Large Viel Erfolg! ✍}
\end{center}

\end{document}
