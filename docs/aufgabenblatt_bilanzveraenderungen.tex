\documentclass[a4paper,11pt]{article}
\usepackage[utf8]{inputenc}
\usepackage[ngerman]{babel}
\usepackage[margin=2.5cm]{geometry}
\usepackage{graphicx}
\usepackage{xcolor}
\usepackage{tcolorbox}
\usepackage{tikz}
\usepackage{booktabs}
\usepackage{array}
\usepackage{multirow}
\usepackage{amsmath}
\usepackage{tabularx}

\definecolor{aktivblau}{RGB}{59, 130, 246}
\definecolor{passivlila}{RGB}{168, 85, 247}

\title{\textbf{Aufgabenblatt: Bilanzveränderungen}}
\author{Berufsschule – Rechnungswesen}
\date{}

\begin{document}

\maketitle

\noindent
\textbf{Name:} \underline{\hspace{6cm}} \hfill \textbf{Datum:} \underline{\hspace{3cm}}

\vspace{0.5cm}

\noindent
\textbf{Hinweis:} Nutze das interaktive Lernspiel, um die Geschäftsvorfälle zu visualisieren und zu verstehen!

\vspace{1cm}

\section*{Aufgabe 1: Bilanzveränderungen erkennen (16 Punkte)}

Bestimme für jeden Geschäftsvorfall die Art der Bilanzveränderung und kreuze an:

\vspace{0.5cm}

\begin{tabularx}{\textwidth}{|p{6.5cm}|X|X|X|X|}
\hline
\textbf{Geschäftsvorfall} & \textbf{AT} & \textbf{PT} & \textbf{BV} & \textbf{BK} \\
\hline\hline
1. Kauf einer Maschine für 50.000 € bar & $\square$ & $\square$ & $\square$ & $\square$ \\
\hline
2. Kunde zahlt Rechnung über 20.000 € & $\square$ & $\square$ & $\square$ & $\square$ \\
\hline
3. Geldabhebung von Bank: 10.000 € auf Kasse & $\square$ & $\square$ & $\square$ & $\square$ \\
\hline
4. Jahresüberschuss 30.000 € in Gewinnrücklage & $\square$ & $\square$ & $\square$ & $\square$ \\
\hline
5. Wareneinkauf auf Rechnung: 40.000 € & $\square$ & $\square$ & $\square$ & $\square$ \\
\hline
6. Kredittilgung bar: 30.000 € & $\square$ & $\square$ & $\square$ & $\square$ \\
\hline
7. Lieferantenverb. durch Bankkredit ablösen: 25.000 € & $\square$ & $\square$ & $\square$ & $\square$ \\
\hline
8. Gesellschafter zahlt Stammkapital ein: 50.000 € & $\square$ & $\square$ & $\square$ & $\square$ \\
\hline
9. Lieferantenrechnung bar bezahlen: 25.000 € & $\square$ & $\square$ & $\square$ & $\square$ \\
\hline
10. Bankkredit umschulden: 15.000 € & $\square$ & $\square$ & $\square$ & $\square$ \\
\hline
11. Maschinenkauf mit Kredit: 60.000 € & $\square$ & $\square$ & $\square$ & $\square$ \\
\hline
12. Privatentnahme: 10.000 € & $\square$ & $\square$ & $\square$ & $\square$ \\
\hline
13. Kauf von Vorräten bar: 15.000 € & $\square$ & $\square$ & $\square$ & $\square$ \\
\hline
14. Rücklage in Stammkapital: 10.000 € & $\square$ & $\square$ & $\square$ & $\square$ \\
\hline
15. Kauf Geschäftsausstattung auf Kredit: 20.000 € & $\square$ & $\square$ & $\square$ & $\square$ \\
\hline
16. Vorräte verkaufen und Kredit tilgen: 15.000 € & $\square$ & $\square$ & $\square$ & $\square$ \\
\hline
\end{tabularx}

\vspace{0.3cm}
\noindent
\textit{Legende: AT = Aktiv-Tausch, PT = Passiv-Tausch, BV = Bilanzverlängerung, BK = Bilanzverkürzung}

\vspace{1cm}

\section*{Aufgabe 2: Bilanzposten zuordnen (8 Punkte)}

Ordne die folgenden Geschäftsvorfälle den betroffenen Bilanzposten zu und gib an, ob diese steigen (+) oder sinken (-):

\vspace{0.5cm}

\textbf{a) Maschinenkauf für 50.000 € in bar}

\vspace{0.3cm}
\noindent
Betroffene Posten (Aktiva): 

1. \underline{\hspace{4cm}} \hspace{0.5cm} Veränderung: \underline{\hspace{2cm}}

2. \underline{\hspace{4cm}} \hspace{0.5cm} Veränderung: \underline{\hspace{2cm}}

\vspace{0.8cm}

\textbf{b) Wareneinkauf auf Rechnung: 40.000 €}

\vspace{0.3cm}
\noindent
Betroffene Posten: 

1. \underline{\hspace{4cm}} (A/P) \hspace{0.5cm} Veränderung: \underline{\hspace{2cm}}

2. \underline{\hspace{4cm}} (A/P) \hspace{0.5cm} Veränderung: \underline{\hspace{2cm}}

\vspace{0.8cm}

\textbf{c) Kredittilgung bar: 30.000 €}

\vspace{0.3cm}
\noindent
Betroffene Posten: 

1. \underline{\hspace{4cm}} (A/P) \hspace{0.5cm} Veränderung: \underline{\hspace{2cm}}

2. \underline{\hspace{4cm}} (A/P) \hspace{0.5cm} Veränderung: \underline{\hspace{2cm}}

\vspace{0.8cm}

\textbf{d) Jahresüberschuss 30.000 € in Gewinnrücklage einstellen}

\vspace{0.3cm}
\noindent
Betroffene Posten (Passiva): 

1. \underline{\hspace{4cm}} \hspace{0.5cm} Veränderung: \underline{\hspace{2cm}}

2. \underline{\hspace{4cm}} \hspace{0.5cm} Veränderung: \underline{\hspace{2cm}}

\newpage

\section*{Aufgabe 3: Bilanzsumme berechnen (6 Punkte)}

Gegeben ist folgende Ausgangsbilanz:

\vspace{0.5cm}

\begin{center}
\begin{tabular}{|l|r||l|r|}
\hline
\multicolumn{2}{|c||}{\textbf{AKTIVA}} & \multicolumn{2}{c|}{\textbf{PASSIVA}} \\
\hline\hline
Maschinen & 100.000 € & Stammkapital & 150.000 € \\
Bankguthaben & 80.000 € & Bankkredit & 50.000 € \\
Vorräte & 50.000 € & Gewinnrücklage & 30.000 € \\
Forderungen & 20.000 € & Verbindlichkeiten & 40.000 € \\
Kasse & 20.000 € &  &  \\
\hline
\textbf{Summe} & \textbf{270.000 €} & \textbf{Summe} & \textbf{270.000 €} \\
\hline
\end{tabular}
\end{center}

\vspace{0.5cm}

Berechne die neue Bilanzsumme nach folgenden Geschäftsvorfällen:

\vspace{0.5cm}

\textbf{a) Maschinenkauf für 50.000 € in bar}

\vspace{0.3cm}
Art der Bilanzveränderung: \underline{\hspace{5cm}}

Neue Bilanzsumme: \underline{\hspace{5cm}}

\vspace{0.8cm}

\textbf{b) Wareneinkauf auf Rechnung: 40.000 €}

\vspace{0.3cm}
Art der Bilanzveränderung: \underline{\hspace{5cm}}

Neue Bilanzsumme: \underline{\hspace{5cm}}

\vspace{0.8cm}

\textbf{c) Kredittilgung bar: 30.000 €}

\vspace{0.3cm}
Art der Bilanzveränderung: \underline{\hspace{5cm}}

Neue Bilanzsumme: \underline{\hspace{5cm}}

\vspace{1cm}

\section*{Aufgabe 4: Bilanz nach Geschäftsvorfällen (10 Punkte)}

Erstelle die Schlussbilanz nach folgenden Geschäftsvorfällen (nutze die Ausgangsbilanz aus Aufgabe 3):

\begin{enumerate}
    \item Kauf von Vorräten bar: 15.000 €
    \item Kunde zahlt Rechnung: 20.000 €
    \item Gesellschafter zahlt Stammkapital ein: 50.000 €
\end{enumerate}

\vspace{0.5cm}

\begin{center}
\begin{tabular}{|l|r||l|r|}
\hline
\multicolumn{2}{|c||}{\textbf{AKTIVA}} & \multicolumn{2}{c|}{\textbf{PASSIVA}} \\
\hline\hline
Maschinen & \underline{\hspace{3cm}} & Stammkapital & \underline{\hspace{3cm}} \\
Bankguthaben & \underline{\hspace{3cm}} & Bankkredit & \underline{\hspace{3cm}} \\
Vorräte & \underline{\hspace{3cm}} & Gewinnrücklage & \underline{\hspace{3cm}} \\
Forderungen & \underline{\hspace{3cm}} & Verbindlichkeiten & \underline{\hspace{3cm}} \\
Kasse & \underline{\hspace{3cm}} &  &  \\
\hline
\textbf{Summe} & \underline{\hspace{3cm}} & \textbf{Summe} & \underline{\hspace{3cm}} \\
\hline
\end{tabular}
\end{center}

\vspace{1cm}

\section*{Aufgabe 5: Transferaufgabe (10 Punkte)}

Erkläre in eigenen Worten, warum bei einem \textbf{Aktiv-Tausch} die Bilanzsumme gleich bleibt, während sie bei einer \textbf{Bilanzverlängerung} steigt.

\vspace{4cm}

\hrule

\vspace{0.5cm}

\noindent
\textbf{Bewertung:} \hspace{2cm} \textbf{Punkte:} \underline{\hspace{1cm}} / 50 \hspace{2cm} \textbf{Note:} \underline{\hspace{2cm}}

\end{document}
