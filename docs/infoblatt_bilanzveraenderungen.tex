\documentclass[a4paper,11pt]{article}
\usepackage[utf8]{inputenc}
\usepackage[ngerman]{babel}
\usepackage[margin=2cm]{geometry}
\usepackage{graphicx}
\usepackage{xcolor}
\usepackage{tcolorbox}
\usepackage{tikz}
\usepackage{booktabs}
\usepackage{array}
\usepackage{multirow}
\usepackage{amsmath}

\definecolor{aktivblau}{RGB}{59, 130, 246}
\definecolor{passivlila}{RGB}{168, 85, 247}
\definecolor{gruen}{RGB}{34, 197, 94}
\definecolor{rot}{RGB}{239, 68, 68}

\title{\textbf{Die 4 Arten von Bilanzveränderungen}}
\author{Infoblatt für Berufsschüler}
\date{}

\begin{document}

\maketitle

\section*{Was ist eine Bilanzveränderung?}

Jeder Geschäftsvorfall in einem Unternehmen verändert die Bilanz. Die Bilanz besteht aus zwei Seiten:
\begin{itemize}
    \item \textcolor{aktivblau}{\textbf{AKTIVA}} (linke Seite): Zeigt, \textit{wie} das Kapital im Unternehmen verwendet wird (Vermögen)
    \item \textcolor{passivlila}{\textbf{PASSIVA}} (rechte Seite): Zeigt, \textit{woher} das Kapital stammt (Kapitalquellen)
\end{itemize}

\vspace{0.5cm}

\textbf{Grundregel:} Aktiva = Passiva (Die Bilanz ist immer ausgeglichen!)

\vspace{1cm}

\section*{Die 4 Arten von Bilanzveränderungen}

\subsection*{1. \textcolor{aktivblau}{Aktiv-Tausch} \hfill ↔️}

\begin{tcolorbox}[colback=blue!5,colframe=aktivblau,title=Definition]
\textbf{Nur die Aktiv-Seite verändert sich}\\
Ein Aktivposten steigt (+), ein anderer sinkt (-)\\
\textbf{Bilanzsumme bleibt gleich}
\end{tcolorbox}

\textbf{Beispiele:}
\begin{itemize}
    \item Maschinenkauf bar (50.000 €): \textcolor{gruen}{+Maschinen}, \textcolor{rot}{-Bank}
    \item Kunde zahlt Rechnung (20.000 €): \textcolor{gruen}{+Bank}, \textcolor{rot}{-Forderungen}
    \item Geld von Bank abheben (10.000 €): \textcolor{gruen}{+Kasse}, \textcolor{rot}{-Bank}
\end{itemize}

\textbf{Merkregel:} \textit{„Tausch auf der Haben-Seite"} – nur Vermögenswerte tauschen sich aus

\vspace{1cm}

\subsection*{2. \textcolor{passivlila}{Passiv-Tausch} \hfill ↔️}

\begin{tcolorbox}[colback=purple!5,colframe=passivlila,title=Definition]
\textbf{Nur die Passiv-Seite verändert sich}\\
Ein Passivposten steigt (+), ein anderer sinkt (-)\\
\textbf{Bilanzsumme bleibt gleich}
\end{tcolorbox}

\textbf{Beispiele:}
\begin{itemize}
    \item Gewinn in Rücklage einstellen (30.000 €): \textcolor{gruen}{+Gewinnrücklage}, \textcolor{rot}{-Jahresüberschuss}
    \item Kredit umschulden (25.000 €): \textcolor{gruen}{+Bankkredit}, \textcolor{rot}{-Verbindlichkeiten}
    \item Rücklage auflösen (10.000 €): \textcolor{gruen}{+Stammkapital}, \textcolor{rot}{-Gewinnrücklage}
\end{itemize}

\textbf{Merkregel:} \textit{„Tausch auf der Soll-Seite"} – nur Kapitalquellen tauschen sich aus

\vspace{1cm}

\subsection*{3. \textcolor{gruen}{Bilanzverlängerung} (Aktiv-Passiv-Mehrung) \hfill 📈}

\begin{tcolorbox}[colback=green!5,colframe=gruen,title=Definition]
\textbf{Beide Seiten steigen gleichzeitig}\\
Aktiva steigt (+) UND Passiva steigt (+)\\
\textbf{Bilanzsumme steigt}
\end{tcolorbox}

\textbf{Beispiele:}
\begin{itemize}
    \item Wareneinkauf auf Rechnung (40.000 €): \textcolor{gruen}{+Vorräte (A)}, \textcolor{gruen}{+Verbindlichkeiten (P)}
    \item Maschinenkauf mit Kredit (60.000 €): \textcolor{gruen}{+Maschinen (A)}, \textcolor{gruen}{+Bankkredit (P)}
    \item Kapitaleinlage (50.000 €): \textcolor{gruen}{+Bank (A)}, \textcolor{gruen}{+Stammkapital (P)}
\end{itemize}

\textbf{Merkregel:} \textit{„Bilanz wird länger"} – beide Seiten wachsen

\vspace{1cm}

\subsection*{4. \textcolor{rot}{Bilanzverkürzung} (Aktiv-Passiv-Minderung) \hfill 📉}

\begin{tcolorbox}[colback=red!5,colframe=rot,title=Definition]
\textbf{Beide Seiten sinken gleichzeitig}\\
Aktiva sinkt (-) UND Passiva sinkt (-)\\
\textbf{Bilanzsumme sinkt}
\end{tcolorbox}

\textbf{Beispiele:}
\begin{itemize}
    \item Kredittilgung bar (30.000 €): \textcolor{rot}{-Bank (A)}, \textcolor{rot}{-Bankkredit (P)}
    \item Rechnung bar bezahlen (25.000 €): \textcolor{rot}{-Kasse (A)}, \textcolor{rot}{-Verbindlichkeiten (P)}
    \item Privatentnahme (10.000 €): \textcolor{rot}{-Kasse (A)}, \textcolor{rot}{-Eigenkapital (P)}
\end{itemize}

\textbf{Merkregel:} \textit{„Bilanz wird kürzer"} – beide Seiten schrumpfen

\vspace{1cm}

\section*{Übersicht: Die 4 Typen auf einen Blick}

\begin{center}
\begin{tabular}{|l|c|c|c|}
\hline
\textbf{Typ} & \textbf{Aktiva} & \textbf{Passiva} & \textbf{Bilanzsumme} \\
\hline\hline
Aktiv-Tausch & {\color{gruen}+} {\color{rot}-} & --- & = \\
\hline
Passiv-Tausch & --- & {\color{gruen}+} {\color{rot}-} & = \\
\hline
Bilanzverlängerung & {\color{gruen}+} & {\color{gruen}+} & {\color{gruen}↑} \\
\hline
Bilanzverkürzung & {\color{rot}-} & {\color{rot}-} & {\color{rot}↓} \\
\hline
\end{tabular}
\end{center}

\vspace{1cm}

\section*{Wichtige Hinweise}

\begin{itemize}
    \item \textbf{Doppelte Buchführung:} Jeder Geschäftsvorfall berührt \underline{mindestens 2 Bilanzposten}
    \item \textbf{Bilanzgleichung:} Aktiva = Passiva gilt \underline{immer} (auch nach Veränderungen!)
    \item \textbf{Keine Mischformen:} Jeder Geschäftsvorfall gehört zu genau einem der 4 Typen
    \item \textbf{Reihenfolge in der Bilanz:} 
    \begin{itemize}
        \item Aktiva: Nach \textit{Liquidität} (von schwer zu leicht liquidierbar)
        \item Passiva: Nach \textit{Fälligkeit} (von langfristig zu kurzfristig)
    \end{itemize}
\end{itemize}

\vspace{1cm}

\section*{Praxis-Tipp}

Um den Typ einer Bilanzveränderung zu bestimmen, frage dich:

\begin{enumerate}
    \item Welche Bilanzseite(n) sind betroffen? (Aktiva, Passiva, beide?)
    \item Steigen oder sinken die Posten? (+/-)
    \item Verändert sich die Bilanzsumme?
\end{enumerate}

\vspace{0.5cm}

\textbf{Beispiel:} „Wir kaufen eine Maschine für 50.000 € und bezahlen bar"

\begin{itemize}
    \item Betroffen: Maschinen (A) und Kasse (A) → nur Aktiva
    \item Maschinen steigen (+50.000 €), Kasse sinkt (-50.000 €)
    \item Bilanzsumme bleibt gleich
    \item \textbf{→ Aktiv-Tausch!}
\end{itemize}

\end{document}
